\documentclass[11pt]{article}

\usepackage{cmap}
\usepackage[T2A]{fontenc}
\usepackage{a4wide}
\usepackage[utf8]{inputenc}
\usepackage[russian]{babel}
\usepackage{vmargin}

\usepackage{graphicx}
\usepackage{subcaption}
\usepackage{float}

\usepackage{color}
\usepackage{indentfirst}

\usepackage{amsmath}
\usepackage{amsthm}
\usepackage{amssymb}
\numberwithin{equation}{section}
%\numberwithin{equation}{subsection}

\setpapersize{A4}
\setmarginsrb{3cm}{2cm}{2cm}{2cm}{0pt}{0mm}{0pt}{13mm}

\theoremstyle{plain}
\newtheorem{thm}{Теорема}[section]
\newtheorem{cor}{Следствие}[section]
\newtheorem{rem}{Замечание}[section]
\newtheorem{problem}{Задача}
\newtheorem{lem}{Утвержение}[section]
\newtheorem{prop}{Свойство}[section]

\theoremstyle{definition}
\newtheorem{defin}{Определение}[section]
\newtheorem{task}{Упражнение}[section]
\newtheorem{exm}{Пример}[section]


\DeclareMathOperator{\expon}{e}
\DeclareMathOperator{\comp}{comp}
\DeclareMathOperator{\rank}{rank}
\DeclareMathOperator{\elip}{\mathcal{E}}
\DeclareMathOperator{\argmin}{argmin}
\DeclareMathOperator{\St}{St}
\DeclareMathOperator{\cl}{cl}


\begin{document}

\setcounter{page}{2}

\tableofcontents
\newpage

\section{Математическая постановка дифференциальной задачи}
В прямоугольной области
\[\Pi = [0,2]\times[0,2]\]
найти дважды гладкую функцию \(u = u(x,y)\) удовлетворяющую уравнению
\begin{equation}
\label{eq}
    -\Delta u = F(x,y), F(x,y) = 2(x^2+y^2)(1-2x^2y^2)\expon^{1-x^2y^2}
\end{equation}
и дополнительному условию
\begin{equation}
\label{BC}
    u(x,y) = \varphi(x,y) = \expon^{1-x^2y^2}
\end{equation}
во всех граничных точках прямоугольника. Оператор Лапласа \(\Delta\) определен равенством:
\[\begin{aligned}
    \Delta u = \frac{\partial^2u}{\partial x^2} + \frac{\partial^2u}{\partial y^2}.
\end{aligned}\]

\section{Численное решение дифференциальной задачи}
Для поиска численного решения задачи~\eqref{eq}-\eqref{BC} используется метод конечных разностей.

В расчётной области \(\Pi\) определим неравномерную прямоугольную сетку
\[\bar w_{h}=\{{(x_i,y_i),\quad i = \overline{0,N_1},\ j=\overline{0,N_2}}\},\]
где
\[\begin{aligned}
x_i = 2f(i/N_1),\quad i = \overline{1,N_1}\\
y_i = 2f(j,N_2),\quad j = \overline{1,N_2}
\end{aligned},\qquad \mbox{здесь }f(t)=\dfrac{(1+t)^{\frac{3}{2}}-1}{2^\frac{3}{2}-1},\quad 0\leqslant t\leqslant 1.
\]
Через \(w_h\) обозначим множество внутренних, а через \(\gamma_h\)~--- множество граничных узлов сетки \(\bar w_{h}.\) Пусть
\[\begin{aligned}
h_i^{(1)} = x_{i+1}-x_{i},\quad i = \overline{1,N_1-1}\\
h_j^{(2)} = y_{i+1}-y_{i},\quad j = \overline{1,N_2-1}
\end{aligned}\]
переменные шаги сетки по оси абсцисс и ординат соответственно. Средние шаги сетки определяются равенствами
\[\begin{aligned}
\bar h_i^{(1)} = \dfrac{h_{i}^{(1)}+h_{i-1}^{(1)}}{2},\quad i = \overline{1,N_1-1}\\
\bar h_j^{(2)} = \dfrac{h_{j}^{(2)}+h_{j-1}^{(2)}}{2},\quad j = \overline{1,N_2-1}
\end{aligned}\]
Рассмотрим линейное пространство \(H\) функций, заданных на сетке \(w_{h}.\) Будем считать, что  пространстве \(H\) задано скалярное произведение и евклидова норма
\[(u,v)=\sum_{i=1}^{N_1-1}\sum_{j=1}^{N_2-1}\bar h_{i}^{(1)}\bar{h}_{j}^{(2)}u(x_{i},y_{i})v(x_i,y_i),\quad\|u\|=\sqrt{(u,u)}.\]
Для аппроксимации уравнения Пуассона~\eqref{eq} воспользуемся пятиточечным разностным оператором Лапласа, который по внутренних узлах сетки определяется равенством:
\[-\Delta_h p_{ij}=\dfrac{1}{\bar{h}_i^{(1)}}\left(\dfrac{p_{ij}-p_{i-1j}}{h_{i-1}^{(1)}}-\dfrac{p_{i+1j}-p_{ij}}{h_{i}^{(1)}}\right)+\dfrac{1}{\bar{h}_{j}^{(2)}}\left(\dfrac{p_{ij}-p_{ij-1}}{h_{j-1}^{(2)}}-\dfrac{p_{ij+1}-p_{ij}}{h_{j}^{(2)}}\right).\]
Здесь предполагается, что функция \(p=p(x_{i},y_{j})\) определена во всех узлах сетки \(\tilde w_{h}.\)
Приближенным решением задачи~\eqref{eq}-\eqref{BC} называется функция \(p=p(x_{i},y_{j}),\) удовлетворяющая уравнениям
\begin{equation}
\label{sheme}
\begin{aligned}
&-\Delta_h p_{ij}=F(x_i,y_j),\quad (x_i,y_j)\in w_{h}\\
&p_{ij}=\varphi(x_{i}y_{i}),\quad (x_{i},y_{j})\in \gamma_{h}.
\end{aligned}
\end{equation}
Эти соотношения представляют собой систему линейных алгебраических уравнений с числом уравнений равным числу неизвестных и определяют единственным образом неизвестные значения \(p_{ij}.\) Совокупность уравнений~\eqref{sheme} называется разностной схемой для задачи~\eqref{eq}-\eqref{BC}.

Приближенное решение системы уравнений~\eqref{sheme} может быть получено итерационным
методом метод скорейшего спуска. В этом методе начальное приближение на границе задается как
\[p_{ij}^{(0)}=\varphi(x_{i},y_{j}),\quad (x_{i},y_{j})\in\gamma_{h},\]
во внутренних узлах сетки \(p_{ij}^{(0)}\) берутся любыми. Метод является одношаговым.
Первая итерация \(p^{(1)}\) вычисляется по формуле
\[p_{ij}^{(1)}=p_{ij}^{(0)}-\tau_{1}r_{ij}^{(0)},\]
где невязка
\begin{equation}
\label{disc}
\begin{aligned}
&r_{ij}^{(k)}=-\Delta_{h}p_{ij}^{(k)}-F(x_i,y_j),\quad (x_{i},y_{j})\in w_{h}\\
&r_{ij}^{(k)}=0,\quad (x_i,y_j)\in\gamma_{h}
\end{aligned}
\end{equation}
а итерационный параметр
\[\tau_{1}=\dfrac{(r^{(0)},r^{(0)})}{(-\Delta_h r^{(0)},r^{(0)})}.\]
Известно, что с увеличением номера итерации \(k\) последовательность сеточных функций \(p^{(k)}\)
сходится к точному решению \(p\) задачи~\eqref{sheme} по норме пространства \(H\), то есть
\[\|p-p^{(k)}\|_{H}\rightarrow 0,\qquad k\rightarrow+\infty. \]
Для последующих итераций используется метод сопряженных градиентов.
Дальнейшие итерации вычисляются по формулам
\[p_{ij}^{(k+1)}=p_{ij}^{(k)}-\tau_{k+1}g_{ij}^{(k)}\quad k=1,2\ldots.\]
Здесь
\[\tau_{k+1}=\dfrac{(r^{(k)},g^{(k)})}{(-\Delta_h g^{(k)},g^{(k)})},\]
вектор
\[\begin{aligned}
&g_{ij}^{k}=r_{ij}^{k}-\alpha_{k}g_{ij}^{(k-1)},\quad k=1,2\ldots,\\
&g_{ij}^{(0)}=r_{ij}^{(0)},
\end{aligned}\]
коэффициент
\[\alpha_{k}=\dfrac{(-\Delta_{h}r^{(k)},g^{k-1})}{(-\Delta_h g^{(k-1)},g^{(k-1)})}\]
Вектор невязки \(r^{(k)}\) вычисляется согласно равенствам~\eqref{disc}. Итерационный процесс останавливается, как только
\[\|p^{(n)}-p^{(n-1)}\|<\varepsilon,\]
где \(\varepsilon\) – заранее выбранное положительное число (в предложенном
варианте задания значение \(\varepsilon=10^{-4}\)).

\section{Постановка задания практикума}
Для задачи~\eqref{eq}-\eqref{BC} требуется найти точное решение и построить
приближённое решение на сетке с числом узлов $N_1 = N_2 = 1000$ и
$N_1 = N_2 = 2000$ и определить погрешность решения по формуле:
\[\begin{aligned}
    \psi=\|u(x,y)-p_{ij}\|.
\end{aligned}\]
Расчеты необходимо проводить на многопроцессорных вычислительных комплексах IBM
Blue Gene/P и <<Ломоносов>>.
Расчеты должны быть проведены для следующего числа процессоров: 128, 256 и 512 на
IBM Plue Gene/P и  8, 16, 32, 64 и 128 на суперкомпьютере <<Ломоносов>>.
Для каждого расчёта определить его
продолжительность и ускорение по сравнению с аналогичным расчётом на одном
вычислительном узле.  При распараллеливании программы необходимо использовать
двумерное разбиение области на подобласти прямоугольной формы, в каждой из
которых отношение $\theta$ количества узлов по ширине и длине должно удовлетворять
неравенствам $\frac{1}{2} \leqslant \theta \leqslant 2$.

На IBM Blue Gene/P также необходимо провести исследование параллельных
характеристик гибридной программы MPI/OpenMP и сравнить полученные результаты с
программой, не используещей директивы OpenMP. Гибридная программа должна
использовать только три процессорных ядра.

\section{Программная реализация}
\subsection{Используемые средства параллелизации}
Для нахождения численного решения поставленной задачи, каждый процессор
производит вычисления на своей области сетки. Для вычисления значений на
границе сетки, необходимо получить от соседних процессоров результаты вычислений
в граничных точках их области, и в свою очередь отправить соседям результаты
своих вычислений. Для обмена массивами данных используются функции
\verb|MPI_Send| и \verb|MPI_Recv|.

При продсчёте скалярного произведения каждый процесс сначала подсчитывает сумму
для своей области, затем с помощью вызова \verb|MPI_Allreduce| с операцией
\verb|MPI_SUM| на каждом процессоре подсчитывается сумма для всей области,
которая используестя в дальнейших вычислениях.

Для подсчёта времени работы программы используется функция \verb|MPI_Wtime|,
перед которой вызывается \verb|MPI_Barrier| для синхронизации процессоров.
В качетсве времени работы программы выбирается максимум по всем процессорам
с помощью вызова \verb|MPI_Allreduce| с операцией \verb|MPI_MAX|.

Возможности технологии OpenMP применяются для распараллеливания циклов
при вычислении значений сеточных функций, оператора Лапласа и элементов
векторов (используется директива
\verb|#pragma omp parallel for|) и при подсчёте скалярного произведения
(директива \verb|#pragma omp parallel for reduction(+:sum)|).

\subsection{Распределение задач между процессорами}
Расчётная область разбивается на прямоугольники, каждый из которых будет
обрабатываться одним из процессоров. Сначала выбирается количество
процессоров по вертикали и горизонтали таким образом, чтобы соотношение
количества процессоров не превышало 2:1 и на каждой стороне их количество
было степенью двойки (это возможно, поскольку по условию задачи исходное
количество процессоров также равно степени двойки). Например, 256 процессоров
распределятся поровну (по 16 на каждую сторону), а при 128 процессорах~---
16 по вертикали и 8 по горизонтали.
Описанное распределение реализовано в функции \verb|distribute_procs|.

Процессоры распределяются последовательно согласно рангу от левого верхнего угла
по строкам. На рис.\,\ref{distr_procs} показан пример распределения 8-ми
процессоров.

\begin{figure}[ht]
    \centering
    \includegraphics[width=0.2\textwidth]{proc_distr.png}
    \caption{Пример распределения процессоров.}
    \label{distr_procs}
\end{figure}

Далее необходимо определить на каждом процессоре, какие точки сетки будут на
нём обрабатываться. Для каждого процессора определяется сегменты индексов
$[x_1, x_2] \subseteq [0, N_1]$ и $[y_1, y_2] \subseteq [0, N_2]$.
На каждом процессоре сохраняются границы этих интервалов.
Точки распределяются по процессорам поровну, остаток распределяестя по
одной точке на процессор. Таким образом, каждый процессор
обрабатывает
либо $\left \lfloor \frac{N_1}{procs} \right \rfloor + 1$, либо
$\left \lfloor \frac{N_1}{procs} \right \rfloor$ точек по оси $Ox$.
Распределение индексов по оси $Oy$ происходит аналогично.
На рис.\,\ref{distr_points} показан пример распределения вдоль одной из
осей для $procs = 256$, $N = 1000$. Тогда на каждую ось приходится 16 процессоров
и между ними необходимо распределить $N + 1 = 1001$ точку. На первые 9
процессоров приходится 63 точки, на оставшиеся 7~--- 62 точки ($62*7 + 63*9 = 1001$).
Распределение реализовано в функции \verb|distribute_points|.

\begin{figure}[ht]
    \centering
    \includegraphics[width=0.2\textwidth]{proc_distr.png}
    \caption{Пример распределения точек по процессорам вдоль одной из осей.}
    \label{distr_points}
\end{figure}

\subsection{Расчёт и обмен данными}

На каждом процессоре после распределения точек происходит инициализация
данных (нулевая итерация), во время которой вычисляются начальные значения
сеточных функций $p$ и $r$. Далее запускается цикл, в котором вычисляется
очередное значение $p$, пока разница между значениями на соседних итерациях
не станет меньше $\varepsilon$. Первая итерация рассчитывается по методу
скорейшего спуска, все последующие~--- методом сопряжённых градиентов.

На каждом процессоре, помимо точек, принадлежащих области, которую обрабатывает
данный процессор, хранятся значение на краницах соседних областей. Эти
значения необходимы при подсчёте оператора Лапласа и скалярного произведения.
После каждой итерации необходимо обменяться граничными значениями с соседними
процессорами (если они есть). Для избежания взаимных блокировок, обмен
происходит в следующем порядке:
\begin{itemize}
    \item получить строку от верхнего соседа;
    \item получить столбец от левого соседа;
    \item отправить свои значения в порядке вниз, направо, вверх, налево;
    \item получить строку от нижнего соседа;
    \item получить столбец от правого соседа.
\end{itemize}

Таким образом, в начале каждой последующей итерации на каждом процессоре
будут находиться актуальные значения сеточной функции $p$ для обрабатываемой
области.

\subsection{Сборка и запуск проекта}

Для компиляции программы использовались команды:
\begin{itemize}
    \item Программа с MPI на суперкомпьютере IBM Blue Gene/P:
        \begin{verbatim}
        mpixlcxx_r -O3 main.cpp -o main
        \end{verbatim}
    \item Гибридная программа MPI/OpenMP на суперкомпьютере IBM Blue Gene/P:
        \begin{verbatim}
        mpixlcxx_r -O3 -openmp main_omp.cpp -o main_omp
        \end{verbatim}
    \item Программа с MPI на суперкомпьютере <<Ломоносов>>:
        \begin{verbatim}
        TODO
        \end{verbatim}
\end{itemize}

Запуск осуществлялся следующим образом:
\begin{itemize}
    \item Программа с MPI на суперкомпьютере IBM Blue Gene/P:
        \begin{verbatim}
        mpisubmit.bg -n PROCS -m smp main -- N1 N2
        \end{verbatim}
    \item Гибридная программа MPI/OpenMP на суперкомпьютере IBM Blue Gene/P:
        \begin{verbatim}
        mpisubmit.bg -n PROCS -m smp -env OMP_NUM_THREADS=3 main -- N1 N2
        \end{verbatim}
    \item Программа с MPI на суперкомпьютере <<Ломоносов>>:
        \begin{verbatim}
        TODO
        \end{verbatim}
\end{itemize}

\newpage
\section{Результаты расчётов}
\begin{table}[h]
\centering
\begin{tabular}{|l|l|l|l|}\hline
Число процессоров $N_p$ & Число точек сетки $N_1 \times N_2$ & Время решения $T$, сек. & Ускорение $S$ \\ \hline
1                       & $1000 \times 1000$                 & ???                     &               \\
128                     & $1000 \times 1000$                 & ???                     &               \\
256                     & $1000 \times 1000$                 & ???                     &               \\
512                     & $1000 \times 1000$                 & ???                     &               \\ \hline
1                       & $2000 \times 2000$                 & ???                     &               \\
128                     & $2000 \times 2000$                 & ???                     &               \\
256                     & $2000 \times 2000$                 & ???                     &               \\
512                     & $2000 \times 2000$                 & ???                     &               \\ \hline
\end{tabular}
    \caption{Результаты работы программы (MPI) на IBM Blue Gene/P.}
\label{tab_mpi}
\end{table}

\begin{table}[h]
\centering
\begin{tabular}{|l|l|l|l|}\hline
Число процессоров $N_p$ & Число точек сетки $N_1 \times N_2$ & Время решения $T$, сек. & Ускорение $S$ \\ \hline
1                       & $1000 \times 1000$                 & ???                     & ???           \\
128                     & $1000 \times 1000$                 & ???                     & ???           \\
256                     & $1000 \times 1000$                 & ???                     & ???           \\
512                     & $1000 \times 1000$                 & ???                     & ???           \\ \hline
1                       & $2000 \times 2000$                 & ???                     & ???           \\
128                     & $2000 \times 2000$                 & ???                     & ???           \\
256                     & $2000 \times 2000$                 & ???                     & ???           \\
512                     & $2000 \times 2000$                 & ???                     & ???           \\ \hline
\end{tabular}
    \caption{Результаты работы программы (MPI/OpenMP) на IBM Blue Gene/P.}
\label{tab_mpi}
\end{table}

\begin{table}[h]
\centering
\begin{tabular}{|l|l|l|l|}\hline
Число процессоров $N_p$ & Число точек сетки $N_1 \times N_2$ & Время решения $T$, сек. & Ускорение $S$ \\ \hline
1                       & $1000 \times 1000$                 & ???                     & ???           \\
8                       & $1000 \times 1000$                 & ???                     & ???           \\
16                      & $1000 \times 1000$                 & ???                     & ???           \\
32                      & $1000 \times 1000$                 & ???                     & ???           \\
128                     & $1000 \times 1000$                 & ???                     & ???           \\ \hline
1                       & $2000 \times 2000$                 & ???                     & ???           \\
8                       & $2000 \times 2000$                 & ???                     & ???           \\
16                      & $2000 \times 2000$                 & ???                     & ???           \\
32                      & $2000 \times 2000$                 & ???                     & ???           \\
128                     & $2000 \times 2000$                 & ???                     & ???           \\ \hline
\end{tabular}
    \caption{Результаты работы программы на СК <<Ломоносов>>.}
\label{tab_mpi}
\end{table}

\section{Сравнение численного решения с аналитическим}
Нетрудно заметить, что функция
\[u(x,y)=\expon^{1-x^2y^2}\]
является точным решением задачи~\eqref{eq}-\eqref{BC} в прямоугольнике \(\Pi.\)
\end{document}

